\documentclass[12pt]{article}
\usepackage{float}
\usepackage{multirow}

\input{structure.tex} 

\title{Preliminary Synthesis of Algebraic Topology for the Final Degree Thesis} % Title of the assignment

\author{Alfonso Mateos Vicente\\ \texttt{alfonso.mateos.vicente@alumnos.upm.es}} % Author name and email address

\date{Universidad Politécnica de Madrid} % University, school and/or department name(s) and a date

%-----------------------------------------------------------------------------

\begin{document}

\maketitle

\newpage
\tableofcontents
\newpage

\section*{Objectives}

In this section we will see the structure of the document and the order in which the topics will be developed on the final thesis.

\begin{enumerate}
	\item Compute trivial zeros of the table of homotopy groups using the Cell Approximation Theorem. 
	\item Study the Freudenthal suspension theorem to see the equalities of the diagonals of the table of homotopy groups.
	\item Show that \(\pi_n(S^n) = Z \ \forall n \in \mathbb{N}\).
	\item Study the long exact sequence of homotopy groups provided by a topological fibration.
	\item Study the Hopf fibration to prove that \(\pi_3(S^2) =Z\) and, thus, the equality of the homotopy groups of \(S^2\) and \(S^3\) from \(\pi_3\).
\end{enumerate}

Consequently, this draft will compile all pertinent concepts, theorems, definitions, and proofs in a concise manner, omitting any supplementary explanatory content. Note that the objectives aren't disposed in the same order as the topics will be developed on the final thesis.

\section{Fundamental group}

This chapter will be the start point to build a basis of the topology theory. The objectives of this chapter are:

\begin{enumerate}
	\item Introduce the concept of \(\pi_1(X, x_0)\).
	\item Prove that \(\pi_1(\mathbb{S}^1) \thickapprox \mathbb{Z}\).
	\item Prove that \(\pi_1(\mathbb{S}^2) = 0\).
	\item Show that the fundamental gorup is not abelian.
\end{enumerate}

\subsection{Definitions and basic concepts}

\begin{definition}[Pointed space \((X, x_0)\)]
	A \textit{pointed space} is a pair \((X, x_0)\) where \(X\) is a topological space and \(x_0 \in X\). The point \(x_0\) is called the \textit{basepoint} and is a distinguised point that will be used as a reference for the study of the space.
\end{definition}

\begin{definition}[Pointed map]
	A \textit{pointed map} between two pointed spaces \((X, x_0)\) and \((Y, y_0)\) is a continuous map \(f: X \rightarrow Y\) such that \(f(x_0) = y_0\).
\end{definition}

\begin{definition}[n-sphere \(\mathbb{S}^n\)]
	The \textit{n-sphere} is the set of points in \(\mathbb{R}^{n+1}\) at a distance \(r\) from the origin, i.e. \(\mathbb{S}^n = \{x \in \mathbb{R}^{n+1} \ | \ ||x|| = r\}\).
\end{definition}

Note that for the purposes of this thesis, the radious of the sphere is not important, so we will consider the unit sphere, i.e. \(\mathbb{S}^n = \{x \in \mathbb{R}^{n+1} \ | \ ||x|| = 1\}\).

\begin{definition}[Path]
	Let \((X, x_0)\) be a pointed space. A \textit{path} in \((X, x_0)\) is a continuous map \(f: [0,1] \rightarrow (X, x_0)\) such that \(f(0) = x_0\).
\end{definition}

\begin{definition}[Loop]
	Let \((X, x_0)\) be a pointed space. A \textit{loop} in \((X, x_0)\) is a continuous map \(f: [0,1] \rightarrow (X, x_0)\) such that \(f(0) = f(1) = x_0\), i.e. a path that starts and ends at the basepoint.
\end{definition}

\begin{definition}[Loop space \(\Omega(X, x_0)\)]
	Let \((X, x_0)\) be a pointed space. The \textit{loop space} of \((X, x_0)\) is the set of all loops in \((X, x_0)\), i.e. \(\Omega(X, x_0) = \{f: [0,1] \rightarrow (X, x_0) \ | \ f(0) = f(1) = x_0\}\).
\end{definition}

\subsection{Homotopy and fundamental group}

\begin{definition}[Homotopy and homotopy of functions]
	Let \((X, x_0)\) and \((Y, y_0)\) be two pointed spaces and let \(f, g: (X, x_0) \rightarrow (Y, y_0)\) be two pointed maps. It is said that \(f\) and \(g\) are \textit{homotopic} if there exists a continuous function \(F: X \times [0,1] \rightarrow Y\) such that \(F(x, 0) = f(x)\), \(F(x, 1) = g(x) \ \forall t \in [0,1]\). The map \(F\) is called a \textit{homotopy} between \(f\) and \(g\).
\end{definition}

\begin{theorem}
	The relation of homotopy \(\sim_H\) is an equivalence relation.
\end{theorem}

\begin{proof}
	Let \(\alpha\), \(\beta\) and \(\gamma\) be three pointed maps between two pointed spaces \((X, x_0)\) and \((Y, y_0)\).
	\begin{enumerate}
		\item \textbf{Reflexivity:} The function \(F: X \times [0,1] \rightarrow Y\) defined as \(F(x, t) = f(x)\) is a homotopy between \(f\) and \(f\).
		\item \textbf{Symmetry:} Let \(F: X \times [0,1] \rightarrow Y\) be a homotopy between \(f\) and \(g\). The function \(G: X \times[0,1] \rightarrow Y\) defined as \(G(x, t) = F(x, 1-t)\) is a homotopy between \(g\) and \(f\).
		\item \textbf{Transitivity:} Let \(F: X \times [0,1] \rightarrow Y\) and \(G: X \times [0,1] \rightarrow Y\) be homotopies between \(f\) and \(g\) and between \(g\) and \(h\) respectively. The function \(H: X \times [0,1] \rightarrow Y\) defined as \(H(x, t)= F(x, 2t)\) if \(t \in [0, 1/2]\) and \(H(x, t) = G(x, 2t-1)\) if \(t \in [1/2, 1]\) is a homotopy between \(f\) and \(h\).
	\end{enumerate}
\end{proof}

\begin{definition}[Fundamental group of \((X, x_0)\), i.e. \(\pi_1(X, x_0)\)]
	Let \((X, x_0)\) a pointed space, the \textit{fundamental group} of \((X, x_0)\) is the set of homotopy classes of loops in \((X, x_0)\), i.e. \(\pi_1(X, x_0) = \Omega(X, x_0) / \sim_H\).
\end{definition}

\begin{definition}[Concatenation of loops]
	Let \((X, x_0)\) be a pointed space and let \(f, g: [0,1] \rightarrow (X, x_0)\) be two loops in \((X, x_0)\). The \textit{concatenation} of \(f\) and \(g\) is the loop \(f \cdot g: [0,1] \rightarrow (X, x_0)\) defined as
	\begin{equation*}
		(f \cdot g)(t) = \begin{cases}
			f(2t) & \text{if } t \in [0, 1/2] \\
			g(2t-1) & \text{if } t \in [1/2, 1]
		\end{cases}
	\end{equation*}
\end{definition}

Now we define a binary operation in the set of homotopy classes of loops in \((X, x_0)\) which will be the group operation of the fundamental group of \((X, x_0)\). Let \([f]\) and \([g]\) be two homotopy classes of loops in \((X, x_0)\). We define the product of \([f]\) and \([g]\) as \([f] \cdot [g] = [f \cdot g]\). We will prove that this operation is well defined.

First of all, let's introduce a very well known lemma that is going to be used in the proof.

\begin{lemma}[Pasting lemma]
	Let \(X\), \(Y\) be both closed (or both open) subsets of a topological space \(Z\) such that \(Z = X \cup Y\) and let \(W\) be also a topological space. If \(f: Z \rightarrow W\) is a function such that \(f|_X\) and \(f|_Y\) are continuous, then \(f\) is continuous.
	\label{lemma:pasting}
\end{lemma}

\begin{theorem}
	Let \((X, x_0)\) be a pointed space. The fundamental group of \((X, x_0)\) endowed by the operation \([f] \cdot [g] = [f \cdot g]\) is a group.
\end{theorem}

\begin{proof}
	Let suppose \(\alpha_1 \sim_H \alpha_2\) and \(\beta_1 \sim_H \beta_2\). We will prove that \(\alpha_1 \cdot \beta_1 \sim_H \alpha_2 \cdot \beta_2\). Let \(F: [0,1] \times [0,1] \rightarrow X\) and \(G: [0,1] \times [0,1] \rightarrow X\) be homotopies between \(\alpha_1\) and \(\alpha_2\) and between \(\beta_1\) and \(\beta_2\) respectively. The function \(H: [0,1] \times [0,1] \rightarrow X\) defined as
	\begin{equation*}
		H(x, t) = \begin{cases}
			F(2s, t) & \text{if } s \in [0, 1/2] \\
			G(1-2s, t) & \text{if } s \in [1/2, 1]
		\end{cases}
	\end{equation*}
	is a continuos because of the pasting lemma. Since
	\begin{equation*}
		H(x, 0) = \begin{cases}
			F(2s, 0) = \alpha_1(2s) & \text{if } s \in [0, 1/2] \\
			G(1-2s, 0) = \beta_1(1-2s) & \text{if } s \in [1/2, 1]
		\end{cases}
	\end{equation*}
	\begin{equation*}
		H(x, 1) = \begin{cases}
			F(2s, 1) = \alpha_2(2s) & \text{if } s \in [0, 1/2] \\
			G(1-2s, 1) = \beta_2(1-2s) & \text{if } s \in [1/2, 1]
		\end{cases}
	\end{equation*}
	we have that \(H(s,0) = (\alpha_1 \cdot \beta_1)(s)\) and \(H(s,1) = (\alpha_2 \cdot \beta_2)(s)\). Therefore, \(\alpha_1 \cdot \beta_1 \sim_H \alpha_2 \cdot \beta_2\). So the operation is well defined, hence \(\pi_1(X, x_0)\) is a group.
\end{proof}

Although the selection of the \(x_0\) is important for the definition of the fundamental group, it turns out that, out of isomorphism, if \(X\) is path-connected, the choice makes no difference. So, from now on, we will denote the fundamental group of \((X, x_0)\) as \(\pi_1(X)\) instead of \(\pi_1(X, x_0)\).

\begin{definition}[Covering space and covering map and evenly covered]
	Let \(X\) be a topological space and let \(p: \tilde{X} \rightarrow X\) be a continuous surjective map. The space \(\tilde{X}\) is called a \textit{covering space} of \(X\) if for every \(x \in X\) there exists an open neighborhood \(U\) of \(x\) such that \(p^{-1}(U)\) is a disjoint union of open sets in \(\tilde{X}\), each of which is mapped homeomorphically onto \(U\) by \(p\). The map \(p\) is called a \textit{covering map}. Such a neighborhood \(U\) is called \textit{evenly covered} by \(p\).
\end{definition}

\begin{definition}[Path's lift]
	Let \(p: \tilde{X} \rightarrow X\) be a covering map and let \(\alpha: [0,1] \rightarrow X\) be a path in \(X\). A \textit{lift} of \(\alpha\) is a path \(\tilde{\alpha}: [0,1] \rightarrow \tilde{X}\) such that \(p \circ \tilde{\alpha} = \alpha\).
\end{definition}

\subsection{Lifting properties}

In this section we will introduce a theorem that will give rise to two lemmas widely used in algebraic topology.

\begin{theorem}
	Given a map \(F : Y \times I \rightarrow X\) and a map \(\tilde{F} : Y \times \{0\} \rightarrow \tilde{X}\) lifting \(F|_{Y \times \{0\}}\), then there is a unique map \(\tilde{F} : Y \times I \rightarrow \tilde{X}\) lifting \(F\) and restrictive to the given \(\tilde{F}\) on \(Y \times \{0\}\). 
	\label{lemma:lift}
\end{theorem}

Note that this is a really general theorem while \(Y\) can be any set we can imagine. This is because next we will introduce two important properties changing this \(Y\).

\begin{proof}
	Let's divide this proof into three sections: the first section has the aim to proof the existence of \(\tilde{F}\), the second one will show that \(\tilde{F}\) is unique if \(Y\) is a point, and the third one will conclude the proof for any \(Y\).

	\textbf{Existence of \(\tilde{F}\):} For a given point \((y_0, t) \in Y \times I\), due to the continuity of \(F\), we can find a neighborhood \(N_t \times (a_t, b_t)\) such that \(F(N_t \times (a_t, b_t))\) is contained in an evenly covered neighborhood of \(F(y_0, t)\). Considering \(A = {N_t \times (a_t, b_t) : t \in I}\), \(A\) covers \({y_0} \times I\). Since \({y_0} \times I\) is compact, we can take a finite subset of \(A\), named \(A'\). This implies we can choose a singles neighborhood \(N\) of \(y_0\) and a partition \(0 = t_0 < t_1 < \dots < t_m = 1\) of \(I\) so that  \(F(N \times [t_i, t_{i+1}])\) is contained in an evenly covered neighborhood of \(F(y_0, t)\), \(U_i\). Let say \(\tilde{F}\) has been constructed on \(N \times [0, t_i]\) starting with the given \(\tilde{F}|_{N \times \{0\}}\). Since \(U_i\) is evenly covered, there exists an open set \(\tilde{U}_i \subset \tilde{X}\) such that \(p(\tilde{U}_i) = U_i\) and containing the point \(\tilde{F}(y_0, t_i)\). Taking a neighborhood of \(y_0\) i.e. \(y_0 \in N_{y_0} \subset N\) we can assume that \(\tilde{F}(N_{y_0}, {t_i})\) is contained in \(\tilde{U}_i\). Now we can define \(\tilde{F}\) on \(N \times [t_i, t_{i+1}]\) to be the composition of \(F\) with the homeomorphism \(p^{-1} : U_i \rightarrow \tilde{U}_i\). After a finite number of steps we eventualy get the desired \(\tilde{F} : N \times I \rightarrow \tilde{X}\) for some neighborhood \(N\) of \(y_0\).
	
	
    \textbf{Uniqueness of \(\tilde{F}\) if \(Y\) is a point:} Suppose two lifts \(\tilde{F}\) and \(\tilde{F}'\) exist, both of which lift \(F : I \rightarrow X\) and agree at the starting point, i.e., \(\tilde{F}(0) = \tilde{F}'(0)\). Choose a partition of \(I\) as before. Assume, inductively, that \(\tilde{F}\) and \(\tilde{F}'\) coincide over \([0, t_i]\). Now consider the interval \([t_i, t_{i+1}]\). Since the image under \(\tilde{F}\) and \(\tilde{F}'\) is connected, \(\tilde{F}([t_i, t_{i+1}])\) and \(\tilde{F}'([t_i, t_{i+1}])\) each lie within a single sheet \(\tilde{U}_i\). As both \(\tilde{F}(t_i)\) and \(\tilde{F}'(t_i)\) are the same by the inductive assumption, they both must lie in the same sheet of \(\tilde{U}_i\). Given that the projection is injective on this sheet, and the two lifts project down to the same map \(F\), it follows that \(\tilde{F}\) and \(\tilde{F}'\) are identical over \([t_i, t_{i+1}]\). Repeating this argument for all such intervals proves the uniqueness of the lift over the entire interval \(I\).
	
	\textbf{Uniqueness of \(\tilde{F}\) over \(Y\):} Finally, the constructed lifts $\tilde{F}$ on sets like $N \times I$ are unique on each segment $\{y\} \times I$ and agree on overlaps, resulting in a well-defined lift $\tilde{F}$ on $Y \times I$. Since $\tilde{F}$ it's continuous on each $N \times I$, therefore, $\tilde{F}$ is continuous using the Lemma \ref{lemma:pasting}. And \(\tilde{F}\) is unique since it is unique on each segment \({y} \times I\).
	\label{theorem:lift}
\end{proof}

This theorem is interesting because now we can directly prove two important lemmas. The first one is called

\begin{lemma}[Path lifting property for covering spaces]
	Given a covering map \(p : \tilde{X} \rightarrow X\), a path \(\alpha : I \rightarrow X\) with \(\alpha(0) = x_0\) and \(\tilde{x}_0 = p^{-1}(x_0)\) then there is a unique lift \(\tilde{\alpha} : I \rightarrow \tilde{X}\) starting at \(\tilde{x}_0\).
	\label{lemma:liftpath}
\end{lemma}

\begin{proof}
	Trivial considering \(Y\) as a point in Theorem \ref{theorem:lift}.
\end{proof}

\begin{lemma}[Homotopy lifting property for covering spaces]
	For each homotopy \(f_t : I \rightarrow X\) of paths starting at a point \(x_0 \in X\) and each \(\tilde{x_0} \in p^{-1}(x_0)\) there is a unique lift \(\tilde{f}_t : I \rightarrow \tilde{X}\) of paths starting at \(\tilde{x}_0\)
	\label{lemma:lifthomotopy}
\end{lemma}

The proof of this one needs a little bit more of work. 

\begin{proof}

\end{proof}

Note that in the title of the lemma, we have written "for covering spaces", this is because in fact the "homotopy lifting property" is a extensively used property in algebraic topology (also called \textit{right lifting property} (RTP) or the \textit{covering homotopy axiom}), but this will be discussed in future chapters.

Finally, we will introduce the lifting criterion which will be used in the future for several proofs.

\begin{theorem}[Lifting criterion]
	Suppose given a covering space \(p : (\tilde{X}, \tilde{x_0}) \rightarrow (X, x_0)\) and a map \(f : (Y, y_0) \rightarrow (X, x_0)\) with \(Y\) path-connected and locally path-conected. Then a lift \(\tilde{f} : (Y, y_0) \rightarrow (\tilde{X}, \tilde{x_0})\) of \(f\) exists if and only if \(f_* (\pi_1(Y, y_0)) \subset p_*(\pi_1(\tilde{X}, \tilde{x_0}))\).
	\label{theorem:criterion}
\end{theorem}

\begin{proof}
	Let's prove the statement in two parts.

	(\(\Rightarrow\)) Assume a lift \(\tilde{f} : (Y, y_0) \rightarrow (\tilde{X}, \tilde{x_0})\) of \(f\) exists. Then for any loop \(\gamma\) in \(Y\) based at \(y_0\), \(f \circ \gamma\) is a loop in \(X\) based at \(x_0\), and its lift starting at \(\tilde{x_0}\) is \(\tilde{f} \circ \gamma\), which is a loop in \(\tilde{X}\) based at \(\tilde{x_0}\). Thus, \(f_*([\gamma]) = [f \circ \gamma]\) is an element of \(p_*(\pi_1(\tilde{X}, \tilde{x_0}))\), since \([f \circ \gamma]\) is the image of \([\tilde{f} \circ \gamma]\) under \(p_*\). This proves that \(f_* (\pi_1(Y, y_0)) \subset p_*(\pi_1(\tilde{X}, \tilde{x_0}))\).

	(\(\Leftarrow\)) Given that \(Y\) is path-connected, for any point \(y \in Y\), there exists a path \(\gamma\) in \(Y\) from \(y_0\) to \(y\). The map \(f\), applied to this path, gives us a path \(f \circ \gamma\) in \(X\) starting at \(x_0\). Because \(p\) is a covering map, there exists a unique lift \(\tilde{f \circ \gamma}\) of this path in \(\tilde{X}\) starting at \(\tilde{x_0}\).
	
	To define the lift \(\tilde{f}\), we set \(\tilde{f}(y)\) to be the endpoint of the lifted path \(\tilde{f \circ \gamma}(1)\). It's necessary to show that this definition of \(\tilde{f}(y)\) does not depend on the choice of path \(\gamma\) in \(Y\) from \(y_0\) to \(y\). 
	
	Suppose \(\gamma'\) is another such path from \(y_0\) to \(y\). Consider the loop \(h_0\) at \(x_0\) formed by traversing \(f \circ \gamma\) and then \(f \circ \gamma'\) in reverse. Because \(f_* (\pi_1(Y, y_0))\) is assumed to be a subset of \(p_*(\pi_1(\tilde{X}, \tilde{x_0}))\), the homotopy class \([h_0]\) has a preimage under \(p_*\), which implies there exists a loop in \(\tilde{X}\) at \(\tilde{x_0}\) that projects to \(h_0\).
	
	By the homotopy lifting property, \(h_0\) can be lifted to a homotopy \(\tilde{h}_t\) in \(\tilde{X}\), with \(\tilde{h}_0\) being the lift of \(h_0\) and \(\tilde{h}_1\) a constant loop at \(\tilde{x_0}\), since it is a lift of a loop in the image of \(p_*\). The lift of \(h_0\) must start with the lift of \(f \circ \gamma'\), follow by the reverse of the lift of \(f \circ \gamma\), and end at the same point, because the end of the lifted path is determined uniquely by its starting point and the homotopy class of the projected path in \(X\).
	
	Continuity of \(\tilde{f}\) can be shown by taking any point \(y \in Y\), choosing a path-connected neighborhood \(V\) of \(y\) small enough that \(f(V)\) lies inside an evenly covered neighborhood \(U\) of \(f(y)\). Because \(p\) restricted to the preimage of \(U\) is a homeomorphism onto \(U\), the lift \(\tilde{f}\) restricted to \(V\) is simply the composition of \(f\) with the inverse of this homeomorphism, hence continuous.
	
	Thus, \(\tilde{f}\) is a well-defined, continuous map lifting \(f\), as required.
\end{proof}

This theorem is a cornerstone in algebraic topology as it provides a necessary and sufficient condition for when a continuous map into a space can be lifted to a map into a covering space of that space. It is widely used in the study of fundamental groups and covering spaces.


\subsection{Fundamental group of the circle}

Now we are prepared to prove one of the most important theorems of the homotopy theory: the fundamental group of the 1-sphere is isomorph to \(\mathbb{Z}\). But, first of all, for illustrative purposes, let's introduce the idea of the proof: Consider \(\mathbb{S}^1\) as the unit circle in the complex plane. Given a loop based on \(x_0 = 1 \in \mathbb{S}^1\) one can imagine how many times the loop winds around the circle, either in positive or negative direction. This "winding number" can be any integer: 0 if the loop does not wind around the circle at all, 1 if it winds once in positive direction, -1 if it winds once in negative direction, 2 if it winds twice in positive direction, etc. This is the idea of the proof, but we will have to formalize it.

\begin{theorem}
	The fundamental group of the 1-sphere is an infinite cyclic group generated by the homotopy class of the loop \(\omega(s) = (cos \ 2\pi s, \ sin \ 2\pi s)\) isomorph to \(\mathbb{Z}\), i.e. \(\pi_1(\mathbb{S}^1) \approx \mathbb{Z}\).
\end{theorem}

\begin{proof}
	Let \(\omega_n : \mathbb{R} \to \mathbb{S}^1\) such that \(\omega_n(t) = e^{2\pi nt}\). The idea is to compare paths in \(\mathbb{S}^1\) with paths in \(\mathbb{R}\) via the map \(p : \mathbb{R} \rightarrow \mathbb{S}^1\) given by \(p(t) = (cos(2\pi n t), \sin(2\pi n t))\). We will prove that \(\pi_1(\mathbb{S}^1) = \{[\omega_n] : n \in \mathbb{Z}\})\) and that \([\omega_n] \neq [\omega_m] \ \ \forall \ n \neq m \in \mathbb{Z}\). Therefore, \(\pi_1(\mathbb{S}^1)\) is isomorph to \(\mathbb{Z}\).

	\textbf{Proof of \(\pi_1(\mathbb{S}^1) = \{[\omega_n] : n \in \mathbb{Z}\}\):} Let \(f : I \to \mathbb{S}^1\) be a loop with basepoint \(x_0 = (1, 0) \in \mathbb{S}^1\). By the Lemma \ref{lemma:liftpath} there exists a unique lift \(\tilde{f} : I \to \mathbb{R}\) of \(f\) such that \(\tilde{f}(0) = 0\). This path \(\tilde{f}\) ends at some integer \(n\) since \(p(\tilde{f}(1)) = f(1) = x_0\) and \(p^{-1}(x_0) = \mathbb{Z}\). Note that since there exists an unique lift, therefore \(\tilde{w}_n \simeq \tilde{f}\). Composing with the projection \(p : \mathbb{R} \to \mathbb{S}^1\) we get a homotopy between \(\omega_n\) and \(f\). So we have that \(\pi_1(\mathbb{S}^1) \subset \{[\omega_n] : n \in \mathbb{Z}\}\). The fact that \(\{[\omega_n] : n \in \mathbb{Z}\} \subset \pi_1(\mathbb{S}^1)\) is trivial because \(\omega_n\) is a loop at the basepoint \(x_0 = 1\). Therefore, \(\pi_1(\mathbb{S}^1) = \{[\omega_n] : n \in \mathbb{Z}\}\).

	\textbf{Proof that \(\omega_n\) is homotopic to \(\omega_m\) if and only if \(n = m\):} Let \(f : I \to \mathbb{S}^1\) which is homotopic to \(\omega_n\) and \(\omega_m\). Let \(f_t\) be a homotopy between \(\omega_n\) and \(\omega_m\). By Lemma \ref{lemma:lifthomotopy} there exists a unique lift \(\tilde{f}_t\) of \(f_t\) such that \(\tilde{f}_t(0) = 0\). The uniqueness part of Lemma \ref{lemma:liftpath} implies that  \(\tilde(f)_0 = \tilde{\omega}_n\) and \(\tilde(f)_0 = \tilde{\omega}_m\), since \(\tilde{f}_t\) is a homotopy of maps, then the endpoint is independent to \(t\), i.e. \(\tilde{f}_t(1) = n\) and \(\tilde{f}_t(1) = m\). Hence \(n = m\).

	Since \(\pi_1(\mathbb{S}^1) = \{[\omega_n] : n \in \mathbb{Z}\}\) and \([\omega_n] \ncong  [\omega_m] \ \forall n, m \in \mathbb{Z}\), therefore \(\pi_1(\mathbb{S}^1)\) is isomorph to \(\mathbb{Z}\).
\end{proof}

\subsection{Seifert-Van Kampen's Theorem}

One of the most important theorems of the algebraic topology is the Seifert-Van Kampen's theorem. This theorem allows us to calculate the fundamental group of a space by using the fundamental groups of its subspaces. In this section we will introduce the Van Kampen's theorem and we will prove that the fundamental group of the 2-sphere is trivial. Also, we will prove that the fundamental group is not necesarilly abelian. Therefore, in this section we will cover the last two objectives of the chapter.

Before delving into the theorem, let's start by illustrating the idea. Let \(A\), \(B\) be two circles intersecting into one unique point \(x_0\). By our preceding calculations we know that \(\pi_1(A) \approx \pi_1(B) \approx \mathbb{S}^1\). So each fundamental group is isomorphic to the cyclic group generated by one element, let denote \(a\), \(b\) as the loops representatives of each cyclic group linked to \(A\), \(B\) respectively. Each product of powers of \(a\) and \(b\) gives a loop around the two circles, i.e. each product of powers is a elements of \(\pi_1(A \cup B)\). Therefore we might correctly think that \(\pi_1(A \cup B) \approx \mathbb{Z} \times \mathbb{Z}\). The Seifert-Van Kampen's theorem is the one that will prove this hypothesis.

Firstly we have to introduce some concepts. 

\begin{definition}[Free product \(\otimes_\alpha G_\alpha\)]
	The free product of a family of groups \( \{ G_\alpha \}_{\alpha \in I} \) is the group consisting of all words of the form 
	\[ g_1^{\epsilon_1} g_2^{\epsilon_2} \cdots g_n^{\epsilon_n} \]
	where for each \( i \), \( g_i \in G_{\alpha(i)} \) and \( \epsilon_i = \pm 1 \) (indicating whether we're taking the group element or its inverse), with the stipulation that if \( g_i \) and \( g_{i+1} \) belong to the same group \( G_\alpha \), then \( g_i g_{i+1} \) is reduced according to the group operation in \( G_\alpha \). This means if \( g_i = e_{\alpha} \) (the identity element in \( G_\alpha \)), then it doesn't appear in the word. Also, if \( g_{i+1} = g_i^{-1} \), then both are omitted.
\end{definition}

The operation in the free product is simply concatenation followed by reduction. For instance, if \( G_\alpha \) and \( G_\beta \) are two groups with the free product \( G_\alpha \otimes G_\beta \), then for \( a, a' \in G_\alpha \) and \( b, b' \in G_\beta \), the product of the word \( ab \) and \( a'b' \) in the free product is \( aba'b' \), and it's reduced only if there's an adjacent pair of elements from the same original group that can be reduced.

First of all, let's recall a theorem frequently applied in the study of Group Theory. Since it is tangential to our current discussion, we will forego its proof

\begin{theorem}[First Isomorphism Theorem]
	Let \(\phi : G \rightarrow G'\) be a group homomorphism. Let \(E\) be the \textit{kernel} of the map \(\phi\). Then \(E \triangleleft G\) and \(G / E \cong im \phi\).
\end{theorem}

Now we are prepared ton introduce the Van Kampen's theorem as follows:

\begin{theorem}[Seifert-Van Kampen's theorem]
	If \(X\) is the union of path-connected open sets \(A_\alpha\) each containing the basepoints \(x_0 \in X\) and if each intersection \(A_\alpha \cap A_\beta\) is path connected, then the homomorphism \(\Phi : \otimes_\alpha \pi_1(A_\alpha) \rightarrow \pi_(X)\) is surjective. If in addition each intersection \(A_\alpha \cap A_\beta \cap A_\gamma\) is path-connected, then the kernel of \(\Phi\) is the normal subgroup \(N\) generated by all elements of form \(i_{\alpha \beta}(\omega) i_{\beta \alpha}(\omega)\) for \(\omega \in \pi_1(A_\alpha \cap A_\beta)\), and hence \(\Phi\) induces an isomorphism \(\pi_1(X) \approx \otimes_\alpha \pi_1(A_\alpha) / N\).
\end{theorem}

\begin{proof}
	
	\textbf{Surjectivity of \(\Phi\):} Given a loop \(f: [0,1] \rightarrow X\) based at \(x_0\), we can cover the image of \(f\) by the open sets \(A_\alpha\). Since \(f\) is compact (as a continuous image of a compact domain), we can find a finite subcover. This means we can express the loop as a product of loops inside individual \(A_\alpha\) and loops in their pairwise intersections. The key insight here is that while traveling within one \(A_\alpha\), we can use the fundamental group of that \(A_\alpha\). But when transitioning from one \(A_\alpha\) to another \(A_\beta\), we can use the fact that their intersection is path-connected to choose a specific path in the intersection to represent this "transition."
	
	\textbf{Kernel of \(\Phi\):} Consider two loops, \(f\) and \(g\), in \(A_\alpha\) and \(A_\beta\) respectively, that represent the same element in \(X\). In other words, their product with a specific path in the intersection is null-homotopic in \(X\). The relation this imposes is that of the loops in \(A_\alpha \cap A_\beta\). Specifically, the relations that come from the intersections look like \(i_{\alpha \beta}(\omega) i_{\beta \alpha}(\omega)^{-1}\), where \(i_{\alpha \beta}\) is the inclusion map and \(\omega\) is a loop in the intersection \(A_\alpha \cap A_\beta\). The requirement that threefold intersections are path-connected ensures that the relations coming from different pairs of \(A_\alpha\) are consistent. Hence, the kernel of \(\Phi\) is precisely the normal subgroup generated by these relations. Using the First Isomorphism Theorem, we get that \(\pi_1(X) \approx \otimes_\alpha \pi_1(A_\alpha) / N\).
\end{proof}

\begin{example}[The fundamental group of two circles intersection in one unique point]
	Let \(A\), \(B\) be two circles and let \({x_0} = A \cap B\). We alredy know that \(\pi_1(A) \approx \pi_1(B) \approx \mathbb{Z}\) and that there exists \(a\), \(b\) such that \(\pi_1(A) = \langle a \rangle\), \(\pi_1(B) = \langle b \rangle\). By the Van Kampen's theorem we get that \(\pi_1(A \cup B) = \langle a, b \rangle\), so \(\pi_1(A \cup B) \approx \mathbb{Z} \times \mathbb{Z}\).
\end{example}

This example shows a counterexample of why the fundamental group is not abelian.


\section{Higher homotopy groups}

At this point we alredy know what is the meaning of the fundamental group and we have proved that the fundamental group of the 1-sphere is isomorph to \(\mathbb{Z}\). But, what about the fundamental group of the 2-sphere? Or the fundamental group of the n-sphere? Is there any other homotopic group than the fundamental group? In this section we will introduce the concept of higher homotopy groups, noted as \(\pi_n(X)\), and we will prove that \(\pi_n(\mathbb{S}^n) = \mathbb{Z} \ \forall n \in \mathbb{N}\), as well as \(\pi_n(\mathbb{S}^m) = 0 \ \forall n > m \in \mathbb{N}\).

We want to naturally generalize the concept of the fundamental group. In the previous chapter we saw that the fundamental group is the set of something that we called "loops" over a equivalence relation that we called "homotopy". Since a loop was a continuous function \(f : I \rightarrow X\) from the interval \([0,1]\) to the space with the particularity that \(f(0) = f(1) = x_0\), the natural generalization of this concept is a continuous function \(f : I^n \rightarrow X\) from the n-dimensional cube to the space with the particularity that \(f(\partial I^n) = x_0\). This is the definition of the n-dimensional loop. Having this in mind we are ready to define the higher homotopy groups.

\begin{definition}[\(n\)-th homotopy group \(\pi_n(X, x_0)\)]
	Let \((X, x_0)\) be a pointed space. Let \(\Psi_n^1(X, x_0)\) be the set of the continuous maps \(\alpha : I^n \rightarrow X\) such that \(\alpha(\delta I^n) = x_0\). Two maps \(\alpha, \beta \in \Psi_n(X, x_0)\) are said to be \textit{homotopic} if there exists a continuous map \(H : I^n \times I \rightarrow X\) such that:
	\[
		\begin{cases}
			H(s, 0) = \alpha(s) & \forall s \in I^n \\
			H(s, 1) = \beta(s) & \forall s \in I^n \\
			H(\delta I^n, t) = x_0 & \forall t \in I
		\end{cases}
	\]
	The set of homotopy classes of \(\Psi_n(X, x_0)\) is called the \(n\)-th homotopy group of \((X, x_0)\) and is denoted by \(\pi_n(X, x_0)\).
\end{definition}

Note that in the case of \(n=1\), the \(n\)-th homotopy group is the fundamental group as \(\partial I = {0,1}\) so \(f(\partial I) = x_0\) is equivalent to \(f(0) = f(1) = x_0\). As we did in the previous sections, let us prove some properties of this set.

\begin{definition}
	Let \((X, x_0)\) be a pointed space. The \textit{concatenation} of two maps \(f, g : I^n \rightarrow X\) is the map \(f + g\) defined as
	\begin{equation*}
		(f + g)(t_1, \dots, t_n) = \begin{cases}
			f(2t_1, t_2, \dots, t_n) & \text{if } t_1 \in [0, 1/2] \\
			g(2t_1-1, t_2, \dots, t_n) & \text{if } t_1 \in [1/2, 1]
		\end{cases}
	\end{equation*}
\end{definition}

The first thought that we might have is that we are changing the notation between the fundamental group, since the "concatenation" defined over the maps of the fundamental groups were defined as \(f \cdot g\), now we are defining it as \(f + g\). But, in fact, this is something that we will clear up in the next theorem. Since only the first coordinate is changing, a lot of arguments of the proof of the theorem will be the same as the ones of the proof of the theorem of the fundamental group. For example, the proof of that the concatenation is well defined is the same as the proof of the theorem of the fundamental group. Also the proof of that \(\pi_n(X, x_0)\) is a group is exactly the same as the proof of the theorem of the fundamental group. However, the fundamental group was not necessary to be abelian, showed in the previous chapter, but if \(n > 1\), it is.

\begin{proposition}
	\(\pi_n(X, x_0)\) is abelian \(\forall n > 1\).
\end{proposition}

\begin{proof}
	Given any two loops \( f, g \) in \( \pi_n(X, x_0) \), we want to show that their concatenation \( f + g \) is homotopic to \( g + f \), i.e., \( f + g \simeq g + f \).

	For the sake of visualization, let's begin by considering the case when \( n = 2 \). This will give us an intuition that can be generalized to higher dimensions.

	Consider the unit square \( I^2 \) where both \( f \) and \( g \) are defined. The product \( f + g \) can be visualized as first traversing the loop \( f \) in the top half of the square and then \( g \) in the bottom half. Similarly, \( g + f \) would involve traversing \( g \) in the top half and \( f \) in the bottom half.

	To prove that these two are homotopic, we'll construct an explicit homotopy between them.

	Start by shrinking the domains of both \( f \) and \( g \) to smaller sub-cubes within \( I^2 \). Ensure that the boundaries of these sub-cubes map to the base point \( x_0 \). After this contraction, we'll have some "free space" within the unit square. Now, with the available free space, we can slide the two sub-cubes around as long as they don't overlap. The key observation here is that we can slide one sub-cube past the other to interchange their positions without them overlapping. Once the positions of the sub-cubes have been interchanged, expand both \( f \) and \( g \) to fill their original domains in \( I^2 \).

	\todo{Image of the process}

	The continuous transformation from \( f + g \) to \( g + f \) through these steps gives us the desired homotopy.

	For higher dimensions, the same principle applies. Instead of considering sub-cubes within a square, we'd be considering sub-hypercubes within a hypercube \( I^n \). The idea of shrinking, sliding, and expanding remains consistent, allowing us to construct a homotopy between \( f \cdot g \) and \( g \cdot f \) in any dimension \( n > 1 \). Hence, \( \pi_n(X, x_0) \) is abelian for all \( n > 1 \).
\end{proof}

Note that the method used for the proof before cannot be applied for the fundamental group as there isn't any space avalaible while trying to shrink the domain.

At this moment, one could have seen that the idea of the loops in higher dimensions are more like "spheres" than something that maps a "square", so now let's introduce a reinterpretation of the definition of the higher homotopy groups in terms of spheres.

\begin{definition}[Higher homotopy groups in terms of spheres]
	Let \((X, x_0)\) be a pointed space. Let \(\Psi_n^2(X, x_0)\) be the set of the continuous maps \(\alpha : \mathbb{S}^n \rightarrow X\) such that \(\alpha(s_0) = x_0\) where \(s_0\) is the basepoint of \(\mathbb{S}^n\), i.e. \(s_0 = (1, 0, 0, ..., 0) \in \mathbb{S}^n\). Two maps \(\alpha, \beta \in \Psi_n(X, x_0)\) are said to be \textit{homotopic} if there exists a continuous map \(H : \mathbb{S}^n \times I \rightarrow X\) such that:
	\[
		\begin{cases}
			H(s, 0) = \alpha(s) & \forall s \in \mathbb{S}^n \\
			H(s, 1) = \beta(s) & \forall s \in \mathbb{S}^n \\
			H(s_0, t) = x_0 & \forall t \in I
		\end{cases}
	\]
	The set of homotopy classes of \(\Psi_n(X, x_0)\) is called the \(n\)-th homotopy group of \((X, x_0)\) and is denoted by \(\pi_n(X, x_0)\).
\end{definition}

These two definitions are equivalent, but the second one is more intuitive. Indeed, we can build a correspondency point-to-point between \(\Psi_n^1(X, x_0)\) and \(\Psi_n^2(X, x_0)\). In this interpretation, the concatenation of two maps \(f, g : \mathbb{S}^n \rightarrow X\) is the composition \(\mathbb{S}^n \rightarrow \mathbb{S}^n \times \mathbb{S}^n \rightarrow X\) where the first map is the diagonal map \(\Delta : \mathbb{S}^n \rightarrow \mathbb{S}^n \times \mathbb{S}^n\) defined as \(\Delta(s) = (s, s)\) and the second map is the product of \(f\) and \(g\), i.e. \((f \times g)(s_1, s_2) = (f(s_1), g(s_2))\). This is the same as the concatenation of the first definition.

\todo{Image of the spheres}

Even the fact that this definition is more difficult at the time of defining the concatenation, it is more intuitive and it is easier to imagine the transformations when working with it.

Other property of the fundamental group that is preserved in higher dimensions is that if \(X\) is path-connected, the choice of the basepoints doesn't matter.

\begin{proposition}
	If \(X\) is path connected, then \(\pi_n(X, x_0) \approx \pi_n(X, x_1) \ \ \forall \ x_0, x_1 \in X\).
\end{proposition}

\begin{proof}
	Let \(x_0, x_1\) be two points of \(X\). As \(X\) is path-connected, we can take a path which starts in \(x_0\) and ends in \(x_1\), \(\gamma : I \rightarrow X\) so that \(\gamma(0) = x_0\) and \(\gamma(1) = x_1\). Now considering each map \(f : I^n \rightarrow (X, x_1)\). We can shrink its domain to a smaller concentri sub-cube in \(I^n\) and insert the path on each radial segment in the space between this sub-cube and the border, therefore doing this, we construct a new path \(\gamma f : I^n \rightarrow (X, x_0)\). Now we have to prove that a homotopy of \(f\) yields a homotopy of \(\gamma f\).

	Let \(g : I^n \rightarrow (X, x_0)\). Using the following formula:

	\[
		h_t(s_1, s_2, ..., s_n) = \begin{cases}
			\gamma (f+0) ((2-t)s_1, s_2, ..., s_n) & \text{if } t_1 \in [0, 1/2] \\
			\gamma (g+0) ((2-t)s_1 + t - 1, s_2, ..., s_n) & \text{if } t_1 \in [1/2, 1]
		\end{cases}
	\]

	Thus we have \(\gamma (f+g) \simeq \gamma (f+0) + \gamma (0+g) \simeq \gamma f + \gamma g\). Defining \(\Psi_\gamma : \pi_n(X,x_1) \rightarrow \pi_n(X, x_0)\) with \(\Psi_\gamma([f]) = [\gamma f]\), the property below shows that \(\Psi_\gamma\) is a homomorphism and the fact that \((\gamma n)f \simeq \gamma (nf)\) and \(1f \simeq f\) imply that \(\Psi_\gamma\) is a isomorphism with the map of the inverse paths of \(\gamma\). Thus \(\pi_n(X, x_0) \approx \pi_n(X, x_1)\).
\end{proof}

With this proposition we can continue noting \(\pi_n(X, x_0)\) as \(\pi_n(X)\) as we did with the fundamental group.

Before continuing, we have to introduce the concept of functor. 

\begin{definition}[Functor]
	A functor is a map between categories. A functor \(F : \mathcal{C} \rightarrow \mathcal{D}\) consists of two components:
	\begin{itemize}
		\item A mapping \(F : Obj(\mathcal{C}) \rightarrow Obj(\mathcal{D})\) that assigns to each object \(X\) of \(\mathcal{C}\) an object \(F(X)\) of \(\mathcal{D}\).
		\item A mapping \(F : Hom(\mathcal{C}) \rightarrow Hom(\mathcal{D})\) that assigns to each morphism \(f : X \rightarrow Y\) of \(\mathcal{C}\) a morphism \(F(f) : F(X) \rightarrow F(Y)\) of \(\mathcal{D}\) such that:
		\begin{itemize}
			\item \(F(id_X) = id_{F(X)}\) for each object \(X\) of \(\mathcal{C}\).
			\item \(F(g \circ f) = F(g) \circ F(f)\) for all morphisms \(f : X \rightarrow Y\) and \(g : Y \rightarrow Z\) of \(\mathcal{C}\).
		\end{itemize}
	\end{itemize}
\end{definition}

\begin{definition}[Induced map]
	Let \(f : X \rightarrow Y\) be a continuous map between two topological spaces. Let \(x_0 \in X\) and \(y_0 = f(x_0)\). The map \(f_* : \pi_n(X, x_0) \rightarrow \pi_n(Y, y_0)\) defined as \(f_*([g]) = [f \circ g]\) is called the \textit{induced map} of \(f\).
\end{definition}

Next we observe the fact that the higher homotopy groups are functors. The first condition of the definition of functor is trivial, so we will only prove the second one. The fact that \(F(id_X) = id_{F(X)}\) for each object \(X\) of \(\mathcal{C}\) is trivial as \(id_{\pi_n(X)}([f]) = [id_{\pi_n(X)} \circ f] = [f]\). The fact that \(F(g \circ f) = F(g) \circ F(f)\) for all morphisms \(f : X \rightarrow Y\) and \(g : Y \rightarrow Z\) of \(\mathcal{C}\) can be proved as \(F(g \circ f)([h]) = [(g \circ f) \circ h] = [g \circ (f \circ h)] = [g \circ F(f)(h)] = F(g)([f \circ h]) = F(g) \circ F(f)([h])\). Hence \(\pi_n\) is a functor.

\subsection{Cell Complexes}

Cell complexes, specifically CW complexes, form a central concept in algebraic topology. They serve as a bridge between the geometric and algebraic realms, allowing for the application of topological methods to study space. In this chapter, we will delve into the fundamental ideas of CW complexes and establish their relation to homotopy groups.

\begin{definition}[CW Complex]
	A CW complex is a type of topological space that is constructed by gluing cells together. Formally, a CW complex \(X\) is built from a discrete set of points \(X^0\), called the 0-cells, by inductively attaching \(n\)-cells \(e^n_\alpha\) via maps \(f_\alpha: S^{n-1} \rightarrow X^{n-1}\) for \(n \geq 1\), where \(X^n\) denotes the \(n\)-skeleton, the space obtained after attaching all cells of dimension \(n\) or less.
\end{definition}

The "C" in CW complex stands for "closure-finite," meaning that each cell is attached to a finite complex, and the "W" stands for "weak topology," which indicates that a set is closed if and only if its intersection with each cell is closed. 

The concept of CW complexes is important because it allows for inductive arguments on the structure of a space. It also simplifies the understanding of homotopy groups, as we can now work with cellular maps that respect the CW structure.

\begin{proposition}
	A covering space projection \(p : (\tilde{X}, \tilde{x}_0) \rightarrow (X, x_0)\) induces a isomorphism \(p_* : \pi_n(\tilde{X}, \tilde{x}_0) \rightarrow \pi_n(X, x_0)\) for all \(n \geq 2\).
\end{proposition}

\begin{proof}
	We will show that \(p_*\) is a bijection whe \(n \geq 2\).

	\textbf{Surjectivity of \(p_*\):} Using the Lifting criterion \ref{theorem:criterion} and the fact that \(\mathbb{S}^n\) is simply-connected for \(n \geq 2\), we get a lift \(\tilde{f}\) of any map \(f : \mathbb{S}^n \rightarrow X\). So given any \([f] \in \pi_n(X, x_0)\), we have that \(p_*([\tilde{f}]) = [f]\). Hence \(p_*\) is surjective.

	\textbf{Injectivity of \(p_*\):} Let \(\tilde{f}, \tilde{g} : \mathbb{S}^n \rightarrow \tilde{X}\) be two maps such that \(p_*([\tilde{f}]) = p_*([\tilde{g}])\). This means that \(p \circ \tilde{f} \simeq p \circ \tilde{g}\). By the homotopy lifting property, we get that \(\tilde{f} \simeq \tilde{g}\). Hence \(p_*\) is injective.

	Therefore \(p_*\) is a bijection and hence an isomorphism.
\end{proof}

\subsection{Cellular Approximation Theorem}

The objective of this section is to prove one of the objectives of the thesis: \(\pi_n(\mathbb{S}^k) = 0 \ \ \forall \ n < k\). First of all, we can define the concept of a map between CW complexes.

\begin{definition}[Cellular map]
	A map \(f : X \rightarrow Y\) between two CW complexes is called \textit{cellular} if it maps each \(n\)-skeleton of \(X\) into the \(n\)-skeleton of \(Y\), i.e. \(f(X^n) \subset Y^n \ \forall n \geq 0\).
\end{definition}

\begin{theorem}[Cellular Approximation Theorem]
	Every map \(f : X \rightarrow Y\) between CW complexes is homotopic to a cellular map. If \(f\) is already cellular on a subcomplex \(A \subset X\), then the homotopy can be chosen to be constant on \(A\).
\end{theorem}

\todo{I'll skip the proof for the moment.}

\begin{corollary}
	\(\pi_n(\mathbb{S}^k) = 0 \ \ \forall n < k\)
\end{corollary}

\begin{proof}
	
	Consider the \(n\)-sphere \(\mathbb{S}^n\) as a CW complex with one 0-cell and one \(n\)-cell. The \(n\)-skeleton of \(\mathbb{S}^n\) is just \(\mathbb{S}^n\) itself since there are no cells of dimension between 0 and \(n\). Now let's take any map \(f: \mathbb{S}^n \rightarrow \mathbb{S}^k\), where \(n < k\).

	According to the Cellular Approximation Theorem, \(f\) is homotopic to a cellular map \(g\). Since \(g\) is cellular and \(n < k\), \(g(\mathbb{S}^n)\) must lie in the \(n\)-skeleton of \(\mathbb{S}^k\). However, the \(n\)-skeleton of \(\mathbb{S}^k\) is just a point (the 0-cell) because there are no cells of dimension \(n\) in \(\mathbb{S}^k\) for \(n < k\). Thus, \(g\) must be a constant map, mapping all of \(\mathbb{S}^n\) to a point in \(\mathbb{S}^k\). Therefore, \(f\) is homotopic to a constant map, and hence \(f\) is null-homotopic. This implies that \(\pi_n(\mathbb{S}^k) = 0 \ \ \forall n < k\).
\end{proof}



\bibliographystyle{plain} 
\bibliography{mybib.bib}


\end{document}
